%%%% MACRO DEFINITION %%%%

\providecommand{\pvivax}{P.~vivax}
\providecommand{\pfalciparum}{P.~falciparum}
\providecommand{\cterm}{C-terminus}
\providecommand{\nterm}{N-terminus}

\providecommand{\e}[1]{\ensuremath{\times 10^{#1}}}
\newcolumntype{P}[1]{>{\centering\arraybackslash}p{#1}}
\newcolumntype{M}[1]{>{\centering\arraybackslash}m{#1}}

\providecommand{\refimage}[1]{\figurename~\ref{fig:#1}}

%TC:macro \note [ignore]



%%%%%%%%%%%%%%%%%%%%%%%%%%%%%%%%%%%%%%%%%%%%%%%%%%%%%%%%%%%%%%%%%%%%%%%%%%%%%%%%%%%%%%%%%%%%%%%%%%%%%%%%%%%%%%%%%%%%%
%%%%%%%%%%%%%%%%%%%%%%%%%%%%%%%%%%%%%%%%%%%%%%%%%%%%%%%%%%%%%%%%%%%%%%%%%%%%%%%%%%%%%%%%%%%%%%%%%%%%%%%%%%%%%%%%%%%%%
%													BEGIN
%%%%%%%%%%%%%%%%%%%%%%%%%%%%%%%%%%%%%%%%%%%%%%%%%%%%%%%%%%%%%%%%%%%%%%%%%%%%%%%%%%%%%%%%%%%%%%%%%%%%%%%%%%%%%%%%%%%%%
%%%%%%%%%%%%%%%%%%%%%%%%%%%%%%%%%%%%%%%%%%%%%%%%%%%%%%%%%%%%%%%%%%%%%%%%%%%%%%%%%%%%%%%%%%%%%%%%%%%%%%%%%%%%%%%%%%%%%

\chapter{Conclusion} % Write in your own chapter title
\label{Chapter6}
\lhead{Chapter 6. \emph{Conclusion}} % Write in your own chapter title to set the page header

This PhD has been an investigation into the properties of high affinity zinc binding sites in proteins, and the extent to which these properties can be used to create predictive models of zinc binding from protein structures or protein sequences.

Investigations of this kind into the nature of zinc binding are of importance at a basic research level, and in a direct medical sense. Understanding the geometric and biochemical properties of zinc binding sites offers insights into how they function --- whether that function is catalytic or structure stabilising. Correspondingly, being able to determine whether a protein binds zinc can offer insights into what the function of that protein is, particularly if the location can be determined too. This is particularly effective in models which can predict from sequence, because entire genomes can be screened in minutes or hours to identify what zinc binding proteins might be present. Medically, the association of pathological zinc binding, particularly in the eye, is a demonstration of how vital a detailed understanding of the mechanisms of zinc binding is to understanding these diseases, and in how predicting the locations of binding sites can be used to aid pharmaceutical interventions.

This project, while an interesting intellectual exercise in and of itself, has been carried out with these crucial benefits to our understanding of the role of zinc in protein functions in health and disease at all times.

Understanding of the properties of zinc binding has been expanded by creating a more comprehensive database of all known zinc binding sites than previous research has created. The key properties and characteristics gleaned from Chapter 4 largely expanded in more detail on some properties that were already commented upon by previous works, albeit at a larger scale.

Probably a much more fundamental contribution to the future of this field, is the existence of ZincBindDB: not as a closed dataset that exists on a PhD Researcher's laptop and which can possibly, after an exchange of emails, be made available to other individuals ---  but rather a web accessible, continually updated resource that has both a web API and web interface. The vast majority of previous research of this kind has resulted in datasets of the former kind, which are not easily shared among researchers and which are, in any case, static once created. The data generated here is very easily accessed by any researcher in the world without the need for my involvement at all, and will continue to be updated with the very latest zinc binding structures indefinitely.

Similarly, those previous works which did result in a web interface have almost all ceased to exist. Even relatively recent papers from the mid-2010s contain URLs which, when accessed in 2021, point to nothing. Nobody can guarantee that they will maintain a service forever --- particularly in the academic sciences where such services are funded by single standalone grants rather than revenue generated by those services. In this case however, all components of ZincBind are open source, cloneable, and reproducible. Even if neither I nor my lab continued to maintain ZincBind (it is currently very much the intention to do so), another researcher could recreate both the services and the data they contained very easily via the GitHub repositories and Docker images \cite{githubdb,dockerdb}, rather than having to recreate from scratch using the Methods section of a paper.

Over the long term, the particular insights into the properties of zinc binding that this PhD has uncovered will be expanded upon and may in some instances be superseded, but the long term presence and availability of this dataset via ZincBind may yet be the most important lasting contribution to the field of zinc binding.

The machine learning component of the PhD has, like the database creation component, built upon previous, similar studies. Here however the difference in methodology has been more stark. From the outset, I deliberately departed from two ubiquitous assumptions that these previous works made: that zinc binding was best detected at the individual residue level, and that the properties of zinc binding sites did not vary much between sub-categories of binding site. By making this tradeoff of only searching for a subset of zinc binding sites, you can achieve superior results in the prediction of zinc binding, and that this tradeoff will become less severe as time passes and the size of the underlying database grows.

Finally, this PhD has resulted in the creation of the Python library atomium for processing PDB structures. This library is already seeing use in the wider field and will continue to be updated with features.

\section{Next Steps}

It is worth examining, in closing, what the logical next steps would be were this project to continue. One very useful extension has already been remarked upon in chapter 5 and indeed will likely happen with very little input from the author or anyone else --- the expansion of the predictive models beyond the existing ten. As chapter 5 explained in detail, there is currently enough data to justify these ten, but insufficient data for any more. As ZincBindDB is automatically updated every week from the Protein Data Bank, over time other families will cross the threshold of data size for these to have predictive models too, and the coverage of the models will gradually increase. This will simply require retraining the models, and adding families to the .dat file used by the build scripts. This will improve the usefulness of the tools over time.

Another possible expansion of the work done here that would have a rather high resultant benefit to effort required ratio, would be to expand beyond just zinc, to other metals. Because the primary objective of the original database creation was to identify properties of zinc binding sites, no previously identified properties were used in creating the database --- the build scripts simply looks for PDB ligands with the name ZN and identifies binding sites for them based on general principles of atomic interactions (sensible atom distance thresholds, minimum angles for atom clashing, etc.). Because nothing about this is specific to zinc, everything done here could just as easily be done for, as an example, copper, or iron. The only restriction used in looking for liganding atoms was the rejection of carbon as a possible liganding atom, but this will be true of all metals. Once this database of all transition metal binding sites were created, likewise creating predictive models for them would also require little to no changes to the already developed codebases. This was not done as part of this PhD to keep it tightly focused on the original research question and not allow for `mission creep' --- but it would greatly increase the usefulness of the already developed code with very little extra work.

An additional possible future avenue for research would be to try and create predictive models of a particular kind of zinc binding site - partial binding sites. These are those binding sites formed of multiple protein chains, in which often the zinc contributes to the oligomerisation of multiple subunits. Predicting these would be particularly useful in the aforementioned pathological aggregation of proteins in the presence of zinc, which is caused by cryptic half-binding sites in the proteins in question. Creating models that detect these is a much more difficult proposition, because it is very difficult to create a correctly labelled dataset. Positive samples are easily enough identified from ZincBindDB by filtering binding sites to include those from multiple chains --- although they are less numerous than the single chain sites used here. Identifying negative samples is much more difficult though. The very nature of these sites means that unless a protein is crystallised in the presence of zinc and in the presence of a protein chain that could make up the other half of a binding site, it will go undetected. There is no easy way to reliably confirm that a pair of residues in a structure could not form a half binding site, simply because it does not in a particular structure, which means the methodology used here would not be suitable. That is not to say there is not path towards doing this, but unlike the previous two proposed future extensions, this could not be done by simply making superficial changes to existing code.

Other suggested features have been made by various interested parties over the course of the PhD --- automatic classification of binding sites as structural or catalytic (sites are not labelled as such in PDB files, but they seem to have very different properties so a dataset could be constructed with some effort), and prediction of binding site affinity (again this might be limited by the amount of available experimental data) to name two that would be of particular use.

Ultimately the best way to decide how future effort would best be spent is to be guided by the actual needs of people using the existing tools. All of the software created as part of this project is open source, and available on GitHub with its associated issue trackers and feature request managers. These offer a much more concrete guide to what would be most useful to the community, and is broadly the approach that will be taken in future. Indeed this decision to make every part of the project open to the community will hopefully be what makes ZincBind long-lasting in the way that its predecessors never were.

Embracing Open Science is the best way to sustain non-profit scientific software --- it is this priniple which has guided this PhD from the beginning, and will continue to guide ZincBind into the future.

