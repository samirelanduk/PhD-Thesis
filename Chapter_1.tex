
%%%% MACRO DEFINITION %%%%

\providecommand{\pvivax}{P.~vivax}
\providecommand{\pfalciparum}{P.~falciparum}
\providecommand{\cterm}{C-terminus}
\providecommand{\nterm}{N-terminus}

\providecommand{\e}[1]{\ensuremath{\times 10^{#1}}}
\newcolumntype{P}[1]{>{\centering\arraybackslash}p{#1}}
\newcolumntype{M}[1]{>{\centering\arraybackslash}m{#1}}

\providecommand{\refimage}[1]{\figurename~\ref{fig:#1}}

% figure that is as wide as the text
\newcommand{\insertfigure}[5][1]{
\begin{figure}[h!]
	\makebox[\textwidth]{\includegraphics[width=#1\textwidth]{Chapter1_pics/#2}}
	\caption[#3]{\small{#4}}\label{fig:#5}
\end{figure}
}

%TC:macro \note [ignore]



%%%%%%%%%%%%%%%%%%%%%%%%%%%%%%%%%%%%%%%%%%%%%%%%%%%%%%%%%%%%%%%%%%%%%%%%%%%%%%%%%%%%%%%%%%%%%%%%%%%%%%%%%%%%%%%%%%%%%
%%%%%%%%%%%%%%%%%%%%%%%%%%%%%%%%%%%%%%%%%%%%%%%%%%%%%%%%%%%%%%%%%%%%%%%%%%%%%%%%%%%%%%%%%%%%%%%%%%%%%%%%%%%%%%%%%%%%%
%													BEGIN
%%%%%%%%%%%%%%%%%%%%%%%%%%%%%%%%%%%%%%%%%%%%%%%%%%%%%%%%%%%%%%%%%%%%%%%%%%%%%%%%%%%%%%%%%%%%%%%%%%%%%%%%%%%%%%%%%%%%%
%%%%%%%%%%%%%%%%%%%%%%%%%%%%%%%%%%%%%%%%%%%%%%%%%%%%%%%%%%%%%%%%%%%%%%%%%%%%%%%%%%%%%%%%%%%%%%%%%%%%%%%%%%%%%%%%%%%%%

\chapter{Introduction} % Write in your own chapter title
\label{Chapter1}
\lhead{Chapter 1. \emph{Introduction}} % Write in your own chapter title to set the page header

Proteins are polymer chains of amino acid residues, and while the chemical diversity of the twenty canonical amino acids utilised in Biology can offer proteins a staggering variety of folds and functionality, there remain chemical processes that are not possible using only this chemical species. Consequently, many proteins use cofactors. These are chemicals which associate with the protein but are not generally covalently bound to them.

In order for a protein to induce this association to happen, it must present a region of its surface that the cofactor in question will experience an attraction to, and for which association will be thermodynamically favourable. The residues that make up this attractive region are called binding sites.

Metal atoms are a common cofactor. In this case the binding site is a metal binding site, and when the metal is zinc we can speak of a `zinc binding site'.

Clearly, zinc will only experience an attraction towards certain kinds of protein surfaces, and so proteins which need to attract a zinc cofactor will have to present surface residues capable of doing this. It should be possible to predict, therefore, whether a protein's structure has a potential zinc binding site on it. Furthermore, because structure is determined by amino acid sequence, it should in principle be possible to predict zinc binding capabilities from protein sequence.

This PhD project is an attempt to develop novel methods for doing so.

\section{Zinc}

Any investigation into the phenomenon of zinc binding to proteins must begin with a look at zinc and its properties.

\subsection{Transition Metals}

Transition metals are often employed as cofactors by proteins, particularly in catalysis. They occupy a different region of electron configuration space than the organic, non-metal atoms that make up the biological amino acids, and so can provide functionality that would otherwise be unavailable.

For example, by definition they have partially filled d-orbitals, which means that they can adopt a number of stable electron numbers each associated with a different charge, known as the different oxidation states. They can all lose their two 4s electrons to become 2+ in charge, but they can usually lose some of their d-electrons too. Iron for example has six d-electrons in its neutral state, and as five d-electrons each in its own orbital is relatively stable, it can also lose one of these d-electrons too, meaning that it can have a charge of 3+ or 2+. These different redox states are very useful when it comes to catalysis, because they can act as temporary custodians of electrons while the reaction substrates are in their intermediate states.

As cations, the transition metals also act as efficient Lewis acids - they can accept lone pairs of electrons. This is also useful in catalysis as it can stabilise intermediate structures by withdrawing excessive negative charge from them.

\note{Also explain how ligand/crystal stabilisation energy works, and breaking of degeneracy}

Unsurprisingly then, many proteins have evolved to take advantage of these useful properties of transition metals by presenting binding sites on their surface that will acquire one of them. This is generally done by bringing residues with available lone pairs into close proximity to each other, in an arrangement that matches the dimensions and orbital geometry of the desired metal.

\subsection{Zinc's Unique Properties}

Though a transition metal in the sense that it ten d-electrons but no 4p electrons, it is an unusual one - to the point where it is often not even classified as one. Many of the properties of transition metals derive from \emph{partially filled} d-orbitals, which zinc does not have. The five fully filled d-orbitals are relatively stable, meaning it can only really be oxidised to 2+ by losing its 4s electrons, so it has just one oxidation state. Electrons cannot be `promoted' from one d-orbital to another in solution as all spaces in the orbitals are already filled, so it has no colour or spectroscopic activity when in solution. At first glance, zinc would appear to be rather unremarkable metal, unworthy of much consideration by evolution.

However, the data show otherwise. About 10\% of all proteins in the human genome are zinc proteins with a zinc binding site - the second most abundant such metal after iron. It is also the only metal found in all classes of enzyme. Clearly, evolution has found zinc very useful.

In fact it is precisely that unremarkableness that has made zinc so attractive. Having just one oxidation state may mean that it cannot hold onto an electron mid-reaction, but it also means that the ion will retain its electronic properties in a wide range of reducing environments - and its status as a particularly good Lewis acid means it is still very useful in catalysis. Its lack of spectroscopic activity may make zinc solutions colourless, but the same arrangement of d-orbitals that causes this also means that there is no energetic penalty for zinc coming out of solution (where it is octahedrally coordinated) and into the binding site \note{Explain this MUCH better}.

\section{Properties of Zinc Binding Sites}

What do zinc binding sites `look like' from a structural point of view? What are their properties at the atomic scale?

These questions were addressed by researchers from the very first crystal structures, and continue to be a topic of research today. The properties of zinc binding sites that distinguish them from other spatially proximate clusterings of residues are ultimately what must be used to predict them, so the advances in this field are crucial to any zinc binding prediction model. The history of our understanding of these properties will be reviewed here.

By the middle of the 1980s, a number of structures had been produced and already a few general themes that would recur over and over again were observed. Catalytic sites tended to have three protein ligands and one water ligand, whereas structural sites generally had four protein residues, for example. One of the first reviews in this area to examine the structures obtained so far identified coordination by sulphur and nitrogen, and tetrahedral geometry, as the two defining characteristics of zinc binding sites \cite{williams1987biochemistry}. With the benefit of hindsight we can see that this is a slight simplification, but it shows that a consensus about `typical features' was beginning to form. Another early review would go on to list most of the basic properties of zinc binding sites we now know - their division into structural and catalytic sites, the near ubiquity of water in catalytic sites, the much stronger preference for tetrahedral geometry in structural sites than in catalytic sites, and the preference for histidine, cysteine, aspartate and glutamate residues \cite{vallee1990zinc}.

These properties would be repeated and elaborated upon by a number of similar reviews in this period \cite{tainer1991metal,vallee1992functional,coleman1992zinc}.

The more structures that were available, the more detailed the inferences that could be made became. A series of reviews looked at typical atom geometries in cysteine binding \cite{chakrabarti1989geometry} and histidine binding \cite{chakrabarti1990geometry}, and while both papers looked at metal binding generally rather than zinc, in both cases the observations were found to be largely metal-agnostic. It was shown that cysteine generally coordinated the metal such that the Zn---S-C-C torsional angle is either 90$^\circ$ or 180$^\circ$ , and that histidine residues generally coordinate the metal via their NE2 nitrogen atom, with the metal lying in the histidyl plane.

Another structural characteristic of zinc binding sites - and indeed metal binding sites generally - was the `hydrophobic contrast' that seemed to exist around them \cite{yamashita1990metal,gregory1993prediction}. Briefly, this is the observation that metal atoms in a protein tend to be surrounded by a shell of hydrophilic atoms (as might be expected), but that these were in turn surrounded by hydrophobic atoms. This could even be implemented as a function of coordinates, which would be at a maximum when centered in such a concentric sphere. It was shown that maxima of this function would cluster around metal binding sites.

Whilst it might be expected that zinc binding sites would have some kinds of structural consensus, it is not immediately obvious that they should have any kind of predictable sequence patterns. However in 1989 it was shown, albeit from a relatively small dataset as existed at that time, that catalytic zinc binding sites seemed to have characteristic ``short and long spacer sequences" \cite{vallee1989short}. That is, the three residues that make up catalytic binding sites seem to be made up of two residues separated by a small stretch of amino acids - around five - and a third residue that is much more distal on the sequence. They proposed that the first two residues were acting as a nucleus, a proto-site that is stabilised by the third residue. This pattern was confirmed by later studies, with the short spacer being found to be generally 2 to 7 residues long \cite{patel2007analysis}.


\section{Predicting Zinc Binding}

There are still many proteins whose function is unknown. In many cases we only know their sequence, though some of these have had their structure solved experimentally.

One means of determining what function a protein might be intended to perform is to look for certain motifs associated with a given function. If, for example, it can be shown that such a protein had one or more zinc binding sites, that could offer insights into what function the protein performs. Indeed, that is ultimately the goal of this project.

Owing to this clear usefulness, there have been a number of attempts to develop such predictors in the past, dating back to the early 1990s. These methods and techniques form the essential context in which my own project should be understood, and also provide the benchmark of success against which my methods should be judged. Here their progress will be reviewed. They can essentially be divided into two broad categories - prediction from structure and prediction from sequence.

\subsection{Prediction from Structure}

Predicting zinc binding from structure takes the atom coordinates of some protein as its input, and produces as output residues in that structure which are believed to be able to bind zinc if zinc were present - generally along with some estimated probability that this should be the case.

This is perhaps of only limited utility, as very often proteins which can bind zinc tightly will have zinc bound when the structure is solved, and so no prediction is required. Where such methods are of use however is when a protein capable of binding zinc is crystallised without zinc being present - presumably because the researchers didn't know it could bind zinc and so did not provide any in the medium. These apoproteins can be scanned by predictors like this to identify unfilled zinc binding sites. \note{NMR invisibility?}

\note{Other uses?}

The earliest attempt at a general purpose prediction algorithm was the Hydrophobic Contrast Function, developed in 1990 \cite{yamashita1990metal} to detect metal binding sites in general. This began from the observation that metal atoms tended to be surrounded by a shell of hydrophilic atoms, which in turn were surrounded by an outer shell of hydrophobic atoms. They developed a function which took as its input a coordinate in space, and returned a measure of this `hydrophobic contrast' at that location. They showed that by scanning every point in the structure on a grid, evaluating the function at every point, and then ranking the points by their score (the function returned values between -1 for inverted contrast and 1 for desired contrast), the highest scoring points would be clustered at sites of metal binding.

This was the first demonstration that a property of metal binding sites could be used to create a predictive model. However it should be noted that it was only tested on structures with metals already present, not on apoproteins, and the initial observation itself was based on a very small number of metal proteins that were available at that time. It was later expanded upon by combining it with a template-based searching algorithm, whereby the known metal binding sites were represented as stripped-down `templates' of just alpha and beta carbons, and used to search proteins using a simple pattern matching algorithm \cite{gregory1993prediction}. Once a set of three residues matching a known template was found, the hydrophobic contrast function was applied to verify that the binding site was a region of high contrast, which it generally was. This was part of a pipeline for engineering binding sites into proteins, so it didn't require the matching residues to be of any particular type, but the general principle of using known site geometry and hydrophobic contrast properties to look in new structures was still in use.

\emph{MetSite}

\emph{Fold-X}

\emph{Goyal and Mande}

\emph{CHED}

\emph{FEATRUE}

\emph{SitePredict}

One of the more recent algorithms for identifying potential zinc binding sites in protein structures is TEMSP \cite{zhao2011structure}. This is actually reminiscent of some of the earlier work on this problem, in that it uses simple geometric patterns of known zinc binding sites - specifically their alpha and beta carbons - to search target structures without any particular machine learning method employed (though there is a certain amount of parameter training). Here the problem is simplified by creating a library of `residue pairs', where for each zinc binding site obtained from the PDB, they store each pairwise combination of liganding residues (actually just certain properties of them, such as alpha-alpha distance etc.) and then search the target protein for these. Once one is found, other template pairs are superimposed on the structure to find additional liganding residues. They used a stricter definition of `true positive' than previous attempts, and still reported a sensitivity of 86.0\% and a selectivity of 95.9\% in their test dataset - barely lower than the values obtained when testing on the training dataset used to refine their parameters. When applying their tool to a dataset of proteins of unknown function, they found some very promising results - finding 186 probable zinc binding sites. \note{So what?} Unfortunately, the online version of this tool is no longer accessible at time of writing.

\subsection{Prediction from Sequence}

%\insertfigure[0.5]{name.eps}{short caption}{long caption}{label}
% N.B. do not \end{document} at the end of the chapters
