
%%%% MACRO DEFINITION %%%%

\providecommand{\pvivax}{P.~vivax}
\providecommand{\pfalciparum}{P.~falciparum}
\providecommand{\cterm}{C-terminus}
\providecommand{\nterm}{N-terminus}

\providecommand{\e}[1]{\ensuremath{\times 10^{#1}}}
\newcolumntype{P}[1]{>{\centering\arraybackslash}p{#1}}
\newcolumntype{M}[1]{>{\centering\arraybackslash}m{#1}}

\providecommand{\refimage}[1]{\figurename~\ref{fig:#1}}

% figure that is as wide as the text
\newcommand{\insertfigure}[5][1]{
\begin{figure}[h!]
	\makebox[\textwidth]{\includegraphics[width=#1\textwidth]{Chapter1_pics/#2}}
	\caption[#3]{\small{#4}}\label{fig:#5}
\end{figure}
}

%TC:macro \note [ignore]



%%%%%%%%%%%%%%%%%%%%%%%%%%%%%%%%%%%%%%%%%%%%%%%%%%%%%%%%%%%%%%%%%%%%%%%%%%%%%%%%%%%%%%%%%%%%%%%%%%%%%%%%%%%%%%%%%%%%%
%%%%%%%%%%%%%%%%%%%%%%%%%%%%%%%%%%%%%%%%%%%%%%%%%%%%%%%%%%%%%%%%%%%%%%%%%%%%%%%%%%%%%%%%%%%%%%%%%%%%%%%%%%%%%%%%%%%%%
%													BEGIN
%%%%%%%%%%%%%%%%%%%%%%%%%%%%%%%%%%%%%%%%%%%%%%%%%%%%%%%%%%%%%%%%%%%%%%%%%%%%%%%%%%%%%%%%%%%%%%%%%%%%%%%%%%%%%%%%%%%%%
%%%%%%%%%%%%%%%%%%%%%%%%%%%%%%%%%%%%%%%%%%%%%%%%%%%%%%%%%%%%%%%%%%%%%%%%%%%%%%%%%%%%%%%%%%%%%%%%%%%%%%%%%%%%%%%%%%%%%

\chapter{Introduction} % Write in your own chapter title
\label{Chapter1}
\lhead{Chapter 1. \emph{Introduction}} % Write in your own chapter title to set the page header

Proteins are polypeptide chains of amino acid residues, and while the chemical diversity of the twenty canonical amino acids utilised in Biology can offer proteins a staggering variety of folds and functionality, there remain chemical processes that are not possible using only this chemical species. Consequently, many proteins use cofactors. These are chemicals which associate with the protein but are not generally covalently bound to them.

In order for a protein to induce this association to happen, it must present a region of its surface that the cofactor in question will experience an attraction to \note{And association must be thermodyamically favourable}. The residues that make up this attractive region are called binding sites.

Metal atoms are a common cofactor \note{Why?}. In this case the binding site is a metal binding site, and when the metal is zinc we can speak of a `zinc binding site'.



%\insertfigure[0.5]{name.eps}{short caption}{long caption}{label}
% N.B. do not \end{document} at the end of the chapters
