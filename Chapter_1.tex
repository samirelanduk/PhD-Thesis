
%%%% MACRO DEFINITION %%%%

\providecommand{\pvivax}{P.~vivax}
\providecommand{\pfalciparum}{P.~falciparum}
\providecommand{\cterm}{C-terminus}
\providecommand{\nterm}{N-terminus}

\providecommand{\e}[1]{\ensuremath{\times 10^{#1}}}
\newcolumntype{P}[1]{>{\centering\arraybackslash}p{#1}}
\newcolumntype{M}[1]{>{\centering\arraybackslash}m{#1}}

\providecommand{\refimage}[1]{\figurename~\ref{fig:#1}}

% figure that is as wide as the text
\newcommand{\insertfigure}[5][1]{
\begin{figure}[h!]
	\makebox[\textwidth]{\includegraphics[width=#1\textwidth]{Chapter1_pics/#2}}
	\caption[#3]{\small{#4}}\label{fig:#5}
\end{figure}
}

%TC:macro \note [ignore]



%%%%%%%%%%%%%%%%%%%%%%%%%%%%%%%%%%%%%%%%%%%%%%%%%%%%%%%%%%%%%%%%%%%%%%%%%%%%%%%%%%%%%%%%%%%%%%%%%%%%%%%%%%%%%%%%%%%%%
%%%%%%%%%%%%%%%%%%%%%%%%%%%%%%%%%%%%%%%%%%%%%%%%%%%%%%%%%%%%%%%%%%%%%%%%%%%%%%%%%%%%%%%%%%%%%%%%%%%%%%%%%%%%%%%%%%%%%
%													BEGIN
%%%%%%%%%%%%%%%%%%%%%%%%%%%%%%%%%%%%%%%%%%%%%%%%%%%%%%%%%%%%%%%%%%%%%%%%%%%%%%%%%%%%%%%%%%%%%%%%%%%%%%%%%%%%%%%%%%%%%
%%%%%%%%%%%%%%%%%%%%%%%%%%%%%%%%%%%%%%%%%%%%%%%%%%%%%%%%%%%%%%%%%%%%%%%%%%%%%%%%%%%%%%%%%%%%%%%%%%%%%%%%%%%%%%%%%%%%%

\chapter{Introduction} % Write in your own chapter title
\label{Chapter1}
\lhead{Chapter 1. \emph{Introduction}} % Write in your own chapter title to set the page header

Proteins are polymer chains of amino acid residues, and while the chemical diversity of the twenty canonical amino acids utilised in Biology can offer proteins a staggering variety of folds and functionality, there remain chemical processes that are not possible using only this chemical species. Consequently, many proteins use cofactors. These are chemicals which associate with the protein but are not generally covalently bound to them.

In order for a protein to induce this association to happen, it must present a region of its surface that the cofactor in question will experience an attraction to, and for which association will be thermodynamically favourable. The residues that make up this attractive region are called binding sites.

Metal atoms are a common cofactor. In this case the binding site is a metal binding site, and when the metal is zinc we can speak of a `zinc binding site'.

Clearly, zinc will only experience an attraction towards certain kinds of protein surfaces, and so proteins which need to attract a zinc cofactor will have to present surface residues capable of doing this. It should be possible to predict, therefore, whether a protein's structure has a potential zinc binding site on it. Furthermore, because structure is determined by amino acid sequence, it should in principle be possible to predict zinc binding capabilities from protein sequence.

This PhD project is an attempt to develop novel methods for doing so.

\section{Zinc}

Any investigation into the phenomenon of zinc binding to proteins must begin with a look at zinc and its properties.

\subsection{Transition Metals}

Transition metals are often employed as cofactors by proteins, particularly in catalysis. They occupy a different region of electron configuration space than the organic, non-metal atoms that make up the biological amino acids, and so can provide functionality that would otherwise be unavailable.

For example, by definition they have partially filled d-orbitals, which means that they can adopt a number of stable electron numbers each associated with a different charge, known as the different oxidation states. They can all lose their two 4s electrons to become 2+ in charge, but they can usually lose some of their d-electrons too. Iron for example has six d-electrons in its neutral state, and as five d-electrons each in its own orbital is relatively stable, it can also lose one of these d-electrons too, meaning that it can have a charge of 3+ or 2+. These different redox states are very useful when it comes to catalysis, because they can act as temporary custodians of electrons while the reaction substrates are in their intermediate states.

As cations, the transition metals also act as efficient Lewis acids - they can accept lone pairs of electrons. This is also useful in catalysis as it can stabilise intermediate structures by withdrawing excessive negative charge from them.

\note{Also explain how ligand/crystal stabilisation energy works, and breaking of degeneracy}

Unsurprisingly then, many proteins have evolved to take advantage of these useful properties of transition metals by presenting binding sites on their surface that will acquire one of them. This is generally done by bringing residues with available lone pairs into close proximity to each other, in an arrangement that matches the dimensions and orbital geometry of the desired metal.

\subsection{Zinc's Unique Properties}

Though a transition metal in the sense that it ten d-electrons but no 4p electrons, it is an unusual one - to the point where it is often not even classified as one. Many of the properties of transition metals derive from \emph{partially filled} d-orbitals, which zinc does not have. The five fully filled d-orbitals are relatively stable, meaning it can only really be oxidised to 2+ by losing its 4s electrons, so it has just one oxidation state. Electrons cannot be `promoted' from one d-orbital to another in solution as all spaces in the orbitals are already filled, so it has no colour or spectroscopic activity when in solution. At first glance, zinc would appear to be rather unremarkable metal, unworthy of much consideration by evolution.

However, the data show otherwise. About 10\% of all proteins in the human genome are zinc proteins with a zinc binding site - the second most abundant such metal after iron. It is also the only metal found in all classes of enzyme. Clearly, evolution has found zinc very useful.

In fact it is precisely that unremarkableness that has made zinc so attractive. Having just one oxidation state may mean that it cannot hold onto an electron mid-reaction, but it also means that the ion will retain its electronic properties in a wide range of reducing environments - and its status as a particularly good Lewis acid means it is still very useful in catalysis. Its lack of spectroscopic activity may make zinc solutions colourless, but the same arrangement of d-orbitals that causes this also means that there is no energetic penalty for zinc coming out of solution (where it is octahedrally coordinated) and into the binding site \note{Explain this MUCH better}.

%\insertfigure[0.5]{name.eps}{short caption}{long caption}{label}
% N.B. do not \end{document} at the end of the chapters
