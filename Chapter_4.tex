%%%% MACRO DEFINITION %%%%

\providecommand{\pvivax}{P.~vivax}
\providecommand{\pfalciparum}{P.~falciparum}
\providecommand{\cterm}{C-terminus}
\providecommand{\nterm}{N-terminus}

\providecommand{\e}[1]{\ensuremath{\times 10^{#1}}}
\newcolumntype{P}[1]{>{\centering\arraybackslash}p{#1}}
\newcolumntype{M}[1]{>{\centering\arraybackslash}m{#1}}

\providecommand{\refimage}[1]{\figurename~\ref{fig:#1}}

%TC:macro \note [ignore]



%%%%%%%%%%%%%%%%%%%%%%%%%%%%%%%%%%%%%%%%%%%%%%%%%%%%%%%%%%%%%%%%%%%%%%%%%%%%%%%%%%%%%%%%%%%%%%%%%%%%%%%%%%%%%%%%%%%%%
%%%%%%%%%%%%%%%%%%%%%%%%%%%%%%%%%%%%%%%%%%%%%%%%%%%%%%%%%%%%%%%%%%%%%%%%%%%%%%%%%%%%%%%%%%%%%%%%%%%%%%%%%%%%%%%%%%%%%
%													BEGIN
%%%%%%%%%%%%%%%%%%%%%%%%%%%%%%%%%%%%%%%%%%%%%%%%%%%%%%%%%%%%%%%%%%%%%%%%%%%%%%%%%%%%%%%%%%%%%%%%%%%%%%%%%%%%%%%%%%%%%
%%%%%%%%%%%%%%%%%%%%%%%%%%%%%%%%%%%%%%%%%%%%%%%%%%%%%%%%%%%%%%%%%%%%%%%%%%%%%%%%%%%%%%%%%%%%%%%%%%%%%%%%%%%%%%%%%%%%%

\chapter{Predicting Zinc Binding Sites in Structures} % Write in your own chapter title
\label{Chapter4}
\lhead{Chapter 4. \emph{Predicting Zinc Binding Sites in Structures}} % Write in your own chapter title to set the page header

The aim of this component of the PhD is to create a system which ultimately is quite simple --- it would take as its input a protein structure, and output a list of zero or more residue combinations which are predicted to form a zinc binding site, with an accompanying probability.

This system is a machine learning system - it uses binary classifiers trained on the large dataset of zinc binding sites described in the previous chapter which `learn' what zinc binding sites look like form that data.

This chapter will describe the approach taken to solving this problem, the models created, the archietcture of the system from the user's point of view, and how the models are accessed and used.

\section{Families}

The first consideration when creating a machine learning classifier is the features that will be used to represent the inputs. Classifiers take their inputs as a vector of numbers and then try to assign the vector to one of two (in the case of binary classifiers) categories. Because the aim here is not just to label an entire protein as either zinc binfing or not zinc binding, but rather to identify the actual zinc binding residues, the proteins themselves are not the inputs - potential zinc binding sites within the protein are.

The main approach is to use combinations of residues as the inputs. That is all plausible combinations of residues will be looked at in turn, turned into a vector, and passed to the model.

Rather than create a single model, the decision was taken early on to create a model for each family. That is, there is a model which looks at sets of three histidines and tries to determine if it is a H3 binding site, a model which looks at sets of four cysteines and tries to determine if it is a C4 binding site, and so on. The main reason for this is that the geometric spacing of residues varies widely between different families, but not within families, which should allow a model to more easily learn what constitutes a likely binding site of a particular family. \note{Definitely need to show this in chapter 3!}

Another advnatage of this approach is a purely practical one. All the possible H3 binding sites in a protein are found by taking all combinations of three histidine residues, and while the combinatorics of this can lead to large numbers of potential sites to check, it is vastly smaller than the combinations of \emph{all} residues that would need to be checked if the model was suppoed to look for a generic zinc binding sites --- an infeasible task for all but the smallest of proteins.

\section{Datasets}
A training dataset was created for each family, using the ZincBindDB API. For each family, all zinc binding sites of that family, belonging to PDBs with a resolution equal to or better than 2 {\AA} were identified --- this quality cutoff being used so that the interatomic distances used in the making of the dataset could be relied upon.

For each site, the PDB file was downloaded.


\section{Old}


Before applying machine learning methods to the dataset, it is worth stopping to see if there are any straightforward identifiable geometric patterns that occur in the zinc binding stuctures gathered, which might be used as templates to search in unknown structures.

There is a wide variety of different binding modes and residue composition, as has been shown, so it is not at first apparent that there should be any usable geometric templates that can be extracted from these. However ZincBind divides the binding sites among families, each with their own residue makeup, and it seems much more plausible that there may be consistent within families. That is, while there may not be a single `zinc binding pattern', there may be a `H3' pattern for example, or a `C2H2' pattern.

Specifically, in this chapter I am interested in patterns of alpha and beta carbons, and the binding sites will be examined in terms of the geometry of these atoms. There are two primary reasons for this:

\begin{enumerate}
   \item The primary use case for searching through protein structures looking for zinc binding patterns is for when the structure is an apo-structure - without zinc. The actual liganding `tips' of liganding residues would be expected to vary more markedly in position than the alpha and beta carbons, which are more constrained by the overall fold of the protein \note{Cite!}.
   \item All residues contain alpha carbons, and all but one contain beta carbons. This means that essentially all residues can be searched for the resultant pattern. While this is not especially useful for the specific use case of finding native zinc binding sites, it \emph{is} useful for zinc binding site engineering. The pattern search could identify all residues which fit a given pattern, and if the residue identities also match it can be flagged as a possible zinc binding site, and if they don't, it can be flagged as a site which could feasibly be mutated to become one.
\end{enumerate}

\subsection{Profile Generation}

The initial step is to go through the ZincBind dataset and create a `profile' for each family. As some of the smaller families are quite esoteric, with large numbers of residues and metals, I decided to limit the scan for the smaller, more highly represented families. Specifically, the initial families were H3 (chosen because carbonic anhydrase is in this family, which is a frequent target of protein engineering), C4 (the most common family), C2H2, and C3H1 (both common families, important in zinc fingers).

Limiting to families which generally ligate a single zinc, rather than coactive binding sites, is important as there would be more constraints on residue position if they are all clustered around a single centre.

For each family, all sites are examined providing (1) their resolution is better than 2.5 ~{\AA}, (2) they only have a single metal, and (3) they ligate using side chain atoms \note{Not at time of writing!}.

The key attributes to determine for any given site are thos eattributes which would be used to actually search a novel structure. This means that the distribution of distances between residue atoms and the metal atom, however interesting, are irrelevant for these purposes because the structures being searched will not have metals present. That leaves the distances between the alpha and beta carbons themselves.

The number of measurements needed for a given binding site will increase with the number of residues, because all combinations need to be considered. For three residue sites such as H3, there are three measurements for the alpha carbons (CA1-CA2, CA2-CA3, CA1-CA3) and three for the beta carbons (CB1-CB2, CB2-CB3, CB1-CB3), for a total of six measurements. For four residue sites, there are six alpha measurements and six beta measurements \note{Give combinations formula explanation}.

This requires a certain consistency in which residues are labelled as 1, 2 etc. In this system, the residues of a given type are ordered by alpha carbon distance to the metal, and numbered accordingly.