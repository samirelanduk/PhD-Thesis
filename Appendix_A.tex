%%%% MACRO DEFINITION %%%%
% if any ...


%%%%%%%%%%%%%%%%%%%%%%%%%%%%%%%%%%%%%%%%%%%%%%%%%%%%%%%%%%%%%%%%%%%%%%%%%%%%%%%%%%%%%%%%%%%%%%%%%%%%%%%%%%%%%%%%%%%%%
%													BEGIN
%%%%%%%%%%%%%%%%%%%%%%%%%%%%%%%%%%%%%%%%%%%%%%%%%%%%%%%%%%%%%%%%%%%%%%%%%%%%%%%%%%%%%%%%%%%%%%%%%%%%%%%%%%%%%%%%%%%%%


\chapter{atomium} % Write in your own chapter title
\label{label_Hi}
\lhead{Appendix. \emph{A}} % Write in your own chapter title to set the page header

%TC:macro \note [ignore]

% write your code here
Much of this project relies very heavily on reading in structure data from files deposited to the Protein Data Bank.

Traditionally these files have been stored and distributed in the form of .pdb files. These are text files that consist of a list of records, with each record being limited to 80 characters. Information is stored in these records at fixed offsets from the start of the line. This file format has a number of limitations - most notably the fact that they cannot store more than 100,000 atoms because only five characters are allocated to atom IDs, and so they cannot go past 99,999.

Consequently, the Protein Data Bank also provides files in the newer .cif file format - and indeed, structures with more than 100,000 atoms are \emph{only} available in this file format.

There are a number of parsers available that can handle these file formats, and which were considered for use in this project - both in Python and otherwise. However, for reasons that will be outlines, it was decided to create a novel Python parser. This parser is called atomium.

\section{Rationale}

\section{Data Structures}

\section{Parsing}

\section{Saving}

\section{Usage}

%%%%%%%%%%%% END %%%%%%%%%%%%
